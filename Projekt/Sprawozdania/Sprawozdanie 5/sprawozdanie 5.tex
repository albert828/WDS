% !TeX encoding = UTF-8
% !TeX spellcheck = pl_PL

% $Id:$

%Author: Wojciech Domski
%Szablon do ząłożeń projektowych, raportu i dokumentacji z steorwników robotów
%Wersja v.1.0.0
%

\newcommand{\kurs}{Wizualizacja danych sensorycznych}
\newcommand{\formakursu}{Projekt}

\newcommand{\doctype}{Za\l{}o\.{z}enia projektowe} %etap I

\newcommand{\projectname}{Wizualizacja samopozycjonującej się platformy fotowoltaicznej}

\newcommand{\osobaA}{Albert \textsc{Lis}, 235534}

\newcommand{\termin}{Śr 17:05 }

\newcommand{\prowadzacy}{dr in\.{z}. Bogdan \textsc{Kreczmer}}

\documentclass[10pt, a4paper]{article}
\usepackage[utf8]{inputenc}
\usepackage{polski}
\usepackage[usenames,dvipsnames]{xcolor}

\include{preambula}
	
\begin{document}

\def\tablename{Tabela}	%zmienienie nazwy tabel z Tablica na Tabela

\begin{titlepage}
	\begin{center}
		\textsc{\LARGE \formakursu}\\[1cm]		
		\textsc{\Large \kurs}\\[0.5cm]		
		\rule{\textwidth}{0.08cm}\\[0.4cm]
		{\huge \bfseries \doctype}\\[1cm]
		{\huge \bfseries \projectname}\\[0.5cm]
		\rule{\textwidth}{0.08cm}\\[1cm]
		
		\begin{flushright} \large
		\osobaA\\[0.4cm]
		
		\emph{Termin: }\termin\\[0.4cm]

		\emph{Prowadzący:} \\
		\prowadzacy \\
		
		\end{flushright}
		
		\vfill
		
		{\large \today}
	\end{center}	
\end{titlepage}

\newpage
\tableofcontents
\newpage

\section{Opis projektu}
\label{sec:OpisProjektu}

Celem projektu jest stworzenie wizualizacji 3D platformy fotowoltanicznej. Platforma jest sterowana za pomocą mikrokontrolera i czterech czujników. Dzięki temu ma możliwość podążania za najintensywniejszym źródłem światła i pozycjonowania się w sposób umożliwiający optymalne korzystanie z energii słonecznej. Dane o pozycji platformy zostaną przesłane do komputera PC. W komputerze zostanie uruchomiona aplikacja pozwalająca pokazywać aktualną pozycję platformy.
\newline
\newline

\begin{figure}[H]
	\centering
	\includegraphics[width=1\textwidth]{figures/diag_uml.png}
	\caption{Architektura systemu}
	\label{fig:Architektura}
\end{figure}

\section{Założenia projektowe}

	\subsection{Komunikacja}
	\begin{enumerate}
		\item Połączenie ze sterownikiem
		\newline
		Realizowane za pomocą modułu Wi-Fi ESP8266 i protokołu UDP/TCP lub bez łączności bezprzewodowej z użyciem portu szeregowego.
		\item Połączenie modułu Wi-Fi z mikrokontrolerem
		\newline
		Realizowane za pomocą portu szeregowego.
	\end{enumerate}

	\subsection{Wizualizacja}
	\begin{enumerate}
		\item Środowisko
		\newline
		Zostanie wykorzystany silnik graficzny UNITY w darmowej wersji.
		\item Modele
		\newline
		Zostaną wygenerowane za pomocą programu Blender.
		\item Tekstury
		\newline
		Zostaną stworzone za pomocą programu GIMP lub pobrane z dowolnej internetowej bazy z darmowymi teksturami.
	\end{enumerate}

\newpage
%Obecne w dokumencie do etapu I
\section{Harmonogram pracy}

	\subsection{Zakres prac}
		\begin{enumerate}
			\item Zapoznanie się ze środowiskiem UNITY
			\newline
			Stworzenie kilku prostych projektów tak aby zapoznać się ze środowiskiem i jego możliwościami.
			
			\item Stworzenie podstawowego modelu 3D
			\newline
			Stworzenie prostego modelu platformy bez dbałości o detale.
			
			\item Implementacja obrotów platformy za pomocą klawiatury
			\newline
			Stworzenie wizualizacji poruszania się modelu za pomocą strzałek na klawiaturze.
			
			\item Stworzenie dokładnych modeli w programie Blender
			\newline
			Stworzenie dokładnego odwzorowania platformy z uwzględnieniem połączeń krawędzi.
			
			\item Wygenerowanie tekstur
			\newline
			Stworzenie lub pobranie z internetu tekstur dla obiektów.
			
			\item Opracowanie standardu komunikacji sterownik - PC
			\newline
			Zastanowienie się nad sposobem przesyłania informacji oraz ich kodowaniem.
			
			\item Implementacja komunikacji sterownik - PC
			\newline
			Implementacja jednostronnej komunikacji między sterownikiem a PC.
			
			\item Implementacja obrotów platformy za pomocą danych ze sterownika
			\newline
			Modyfikacja istniejącego sterowania w taki sposób aby zwizualizowany stan platformy zgadzał się z rzeczywistym.
			
			\item Poprawki stylistyczne
			\newline
			Poprawa elementów które okazały się niedopracowane w trakcie projektu.
		\end{enumerate}
	
	\subsection{Kamienie milowe}
		\begin{enumerate}
			\item Implementacja działającej wizualizacji w oparciu o sterowanie klawiaturą
			\item Implementacja poprawnej komunikacji sterownik - PC
			\item Implementacja wizualizacji w oparciu o dane ze sterownika
		\end{enumerate}
	
	\subsection{Wykres Gantta}
	
	\begin{figure}[H]
		\centering
		\includegraphics[width=1.4\textwidth, angle = 90]{figures/harm1.png}
		\caption{Diagram Gantta}
		\label{fig:DiagramGantta}
	\end{figure}

\section{Projekt interfejsu graficznego}
	\subsection{Funkcjonalność UI}
		\begin{enumerate}
			\item Lista wyboru nazwy portu szeregowego
				\newline
				Powinna umożliwić wybranie portu do którego podłączony jest sterownik.
			\item Lista wyboru prędkości połączenia
				\newline
				Powinna zawierać takie prędkości wyrażone w bodach względem których przesłanie pakietu danych będzie trwało mniej niż 1/60[s]. 
			\item Przycisk nawiązania połączenia
				\newline
				Przycisk umożliwiający nawiązanie połączenia ze sterownikiem po wybraniu odpowiednich parametrów.
			\item Przycisk zakończenia połączenia
				\newline
				Przycisk umożliwiający zakończenie połączenia ze sterownikiem.
			\item Przycisk zamknięcia aplikacji
				\newline
				Przycisk umożliwiający zamknięcie aplikacji. Powinien realizować również akcję zamykania połączenia jeśli nadal by było ono otwarte.	
		\end{enumerate}
	
	\subsection{Funkcjonalność aplikacji}
		\begin{enumerate}
			\item Wyświetlanie aktualnej wartości natężenia światła
				\newline	
				Wartość natężenia światła powinna być wyświetlana nad platformą np w postaci napisu 3D.
			\item Wyświetlanie aktualnej pozycji
				\newline	
				Powyżej/poniżej wartości natężenia światła powinna być wyświetlana informacja o aktualnej pozycji. Za pomocą schematu XYZ.
		\end{enumerate}
	
	\subsection{Graficzna reprezentacja aplikacji}
		\begin{enumerate}[I.]
			\item Schemat
				\begin{figure}[H]
					\centering
					\includegraphics[width=1\textwidth]{figures/paint.png}
					\caption{Wygląd aplikacji}
					\label{fig:ArchitekturaBD1}
				\end{figure}
			\newpage
			\item Szczegółowy opis UI
				\begin{enumerate}[1]
					\item COM
						\newline	
						Lista modyfikująca parametr odpowiedzialny za nazwę portu w skrypcie obsługi portu szeregowego.
					
					\item Baud Rate
						\newline	
						Lista modyfikująca parametr odpowiedzialny za prędkość transmisji w skrypcie obsługi portu szeregowego.
					
					\item Connect
						\newline	
						Przycisk wywołujący funkcję odpowiedzialną za nawiązanie połączenia w skrypcie obsługi portu szeregowego.
						
					\item Disconnect
						\newline	
						Przycisk wywołujący funkcję odpowiedzialną za zamknięcie połączenia w skrypcie obsługi portu szeregowego.
						
					\item Close application
						\newline	
						Przycisk wywołujący skrypt odpowiedzialny za zamknięcie aplikacji.
				\end{enumerate}
			\item Szczegółowy opis napisów interaktywnych
			\begin{enumerate}[a)]
				\item Value
				\newline	
				Napis wyświetlający bieżącą wartość natężenia światła. Połączony ze skryptem rotacji aby dostosowywał swoje położenie względem kamery.
				
				\item X \dots Y \dots Z \dots
				\newline	
				Napis wyświetlający aktualne położenie we współrzędnych kartezjańskich. Połączony ze skryptem rotacji aby dostosowywał swoje położenie względem kamery.
			\end{enumerate}
		\end{enumerate}
			
\newpage
\section{Wstępne rezultaty}
		\subsection{Zmiany w projekcie}
		Nastąpiła zmiana środowiska programistycznego z Unity na Qt + OpenGL. To pociągnęło za sobą zmiany w harmonogramie pracy i podejście do projektu. Najpierw zostanie stworzona komunikacja między PC a sterownikiem a następnie wizualizacja 3D. Dodatkowo zmieni się format przesyłanych danych. Aktualnie przewiduję że pakiet danych będzie wyglądał następująco: 
		\begin{center}
			"H\footnote{rotacja horyzontalna} ... V\footnote{rotacja wertykalna} ... L\footnote{natężenie światła} ... I\footnote{Zmierzone natężenie prądu} ... CRC\footnote{32-bitowa suma kontrolna} ... $\backslash$n\footnote{znak końca pakietu danych}".
		\end{center}
		Gdzie ... to poszczególne wartości. Natomiast separator to znak spacji.
		
		\subsection{Zrealizowane zadania}
			\begin{enumerate}
				\item \textbf{Graficzy interfejs użytkownika} \newline
				Stworzyłem aplikację komunikującą się ze sterownikiem za pomocą portu szeregowego. Aplikacja po uruchomieniu prosi o podanie prędkości komunikacji w bodach oraz portu do którego został przyłączony sterownik. Domyślna prędkość to 9600 bodów. Natomiast lista portów wczytuje tylko te dostępne.
				\begin{figure}[H]
					\centering
					\includegraphics[width=0.3\textwidth]{figures/opcje.png}
					\caption{Okno opcji}
					\label{fig:opcje}
				\end{figure}
				
				Po zaakceptowaniu ustawień uruchamia się okno główne w którym mamy opcje Connect oraz Disconnect. Obie wzajemnie się wykluczają. Dodatkowo gdy połączenie jest aktywne wygaszona zostaje opcja zmiany ustawień połączenia.
				
				\begin{figure}[H]
					\centering
					\includegraphics[width=0.4\textwidth]{figures/okno_glowne.png}
					\caption{Okno główne}
					\label{fig:okno_glowne}
				\end{figure}
			
				\item \textbf{Komunikacja} \newline
				Komunikacja jest uruchamiana w osobnym wątku tak aby nie zakłócać pracy głównego okna. Port szeregowy został skonfigurowany z 8 bitami danych, bitem parzystości oraz bitem stopu. Aktualnie przesyłane dane wyświetlam za pomocą konsoli. Gdy suma kontrolna się nie zgadza wyświetlam komunikat o niepoprawnej ramce danych.
			\end{enumerate}
		
\section{Zaawansowane rezultaty}
	\subsection{Komunikacja}
	Zmianie uległ format danych. Zostały dodane dodatkowe pola. Oraz liczona jest suma kontrolna 8-bitowa. Nazwa sumy: CRC-8-Dallas/Maxim, Wielomian: 0x8C.
		\begin{center}
			"H\footnote{rotacja horyzontalna}... V\footnote{rotacja wertykalna}... L\footnote{natężenie światła}... U\footnote{Napięcie}... I\footnote{Prąd}... P\footnote{Moc}... CRC\footnote{8-bitowa suma kontrolna}... $\backslash$r$\backslash$n\footnote{znak końca pakietu danych}".
		\end{center}
	Przykładowe wyniki:
		\begin{figure}[H]
			\centering
			\includegraphics[width=0.6\textwidth]{figures/serial_wyniki.png}
			\caption{Przykładowe wyniki}
			\label{fig:serial_wyniki}
		\end{figure}
	
	\subsection{Graficzny interfejs użytkownika}
		\begin{enumerate}
			\item Okno główne
				\begin{figure}[H]
					\centering
					\includegraphics[width=0.98\textwidth]{figures/okno_glowne1.png}
					\caption{Okno główne}
					\label{fig:okno_glowne1}
				\end{figure}
			Po lewej znajdują się wykresy wartości pomiarowych. Natomiast po prawej stronie mamy wizualizację platformy fotowoltaicznej.
					
			\item Wykresy
				\begin{figure}[H]
					\centering
					\includegraphics[width=1.0\textwidth]{figures/wykresy.png}
					\caption{Wykresy}
					\label{fig:wykresy}
				\end{figure}
			Do rysowania wykresów używam biblioteki zewnętrznej QCustomPlot. Na osi pionowej znajdują się poszczególne wartości pomiarów. Na osi poziomej mamy czas w formacie minuty:sekundy. Wykres jest dynamiczny o stałym oknie czasowym. Na przykładowym rysunku \ref{fig:wykresy} wynosi ono 8 sekund.
			\newpage
			\item Wizualizacja
				\begin{figure}[H]
					\centering
					\includegraphics[width=1.0\textwidth]{figures/wizualizacja.png}
					\caption{Wizualizacja}
					\label{fig:wizualizacja}
				\end{figure}
			Zaimplementowałem dwuosiową rotację. Horyzontalna (cała platforma) oraz wertykalna (tylko panel fotowoltaiczny).
		\end{enumerate}
\section{Rezultaty prawie końcowe}
	\subsection{Zrealizowane zadania}
	Ze względu na wysoki poziom zaawansowania projektu ograniczyłem się do podstawowych zadań, a dostępne zasoby, wykorzystałem do realizacji projektów równoległych. \\
	Zrealizowane zadania:
		\begin{enumerate}
			\item  Stworzenie dokumentacji za pomocą generatora Doxygen.
			\item Dopracowanie komunikacji między sterownikiem a komputerem.
			\begin{itemize}
				\item Obliczenie ilości półkroków silnika krokowego \\ Ilość półkroków jakie wykonuje silnik w czasie pełnego obrotu wynosi 2734.
				\item Wykorzystanie rzeczywistych pomiarów położenia horyzontalnego i wertykalnego.
				\item Wykorzystanie rzeczywistych pomiarów natężenia światła.
				\item Wykorzystanie dzielnika napięciowego do pomiaru przepływu prądu oraz napięcia.
			\end{itemize}
			\item Testy poprawności modyfikacji.
		\end{enumerate}
	
	\subsection{Zmiany w projekcie}
	Niestety w trakcie transportu nieodwracalnemu uszkodzeniu uległ panel fotowoltaiczny, co dyskwalifikuje jego użycie w projekcie. Na jego miejsce zostanie użyty fotorezystor, który będzie symulował jego działanie. Po pierwszych testach i wprowadzeniu współczynników korekcyjnych rozwiązanie działa poprawnie.
	
\end{document}







































